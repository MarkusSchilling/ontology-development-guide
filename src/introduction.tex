\section{Introduction}

\subsection{Why does this document exist?}

Ontology development poses a formidable challenge due to its inherent complexity arising from the need to capture and represent diverse knowledge domains in a structured and coherent manner. The difficulty arises from the ambiguity of natural language, the dynamic nature of information, and the intricate relationships between concepts. Additionally, reconciling differing perspectives and ensuring ontological relevance across various applications further complicates the task. Achieving a balance between specificity and generality while accommodating evolving knowledge landscapes adds another layer of intricacy to ontology development. Thus, the multifaceted nature of ontology creation underscores its inherent difficulty in providing a comprehensive and accurate representation of complex real-world phenomena.

Additionally, there exist many resources that convey a picture of what an ontology is that is detremental to the development on ontologies that are useful and accepted by a wider range of experts. 

\subsection{Introductory literature}

The following workflow is largely inspired by an infrastructure described by Allemang et. al at FOIS 2021 \cite{kendall-workflow} and the methods used to develop large ontologies such as (FIBO) \cite{fibo2013} and the Open Energy Ontology (OEO) \cite{oeo2021}.

\begin{itemize}
    \item Maria Keet: \cite{keet2018introduction}.
    \item Neuhaus et al.: \cite{Neuhaus2022OntologyDI}
    \item Smith et al.: \cite{bfo-book}
\end{itemize}

Furthermore, as introduction to the collaborative project \href{http://www.materialdigital.de}{Innovation Platform MaterialDigital (PMD)}, in the frame of which this document was created, and to the process of ontology development, the following publications are supposed to provide more information: 

\begin{itemize}
    \item Bayerlein et al., \href{https://doi.org/10.1002/adem.202101176}{A Perspective on Digital Knowledge Representation in Materials Science and Engineering}: \cite{BayerleinPerspective2022}.
    \item Bayerlein et al., \href{https://doi.org/10.1016/j.matdes.2023.112603}{PMD Core Ontology: Achieving semantic interoperability in materials science}: \cite{BayerleinPMDco2024}
    \item Schilling et al., FAIR and structured data: A domain ontology aligned with standard-compliant tensile testing: \cite{SchillingTTO2023}
\end{itemize}