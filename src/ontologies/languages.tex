\subsection{The Web Ontology Language}
\label{ssec:ontology-languages}

The Web Ontology Language (OWL) is the most widely used formal language for ontology development. Our introduction of ontologies aligns closely with the structure of OWL. The main building blocks of OWL are logical axioms that define classes and their relationships. The OWL specification defines a number formats that can be used to develop and exchange ontologies. The default exchange format for OWL is its RDF/XML syntax. This syntax is however rather unwieldy and rather hard to read for most people. In this guide, we will therefore use the more readable OWL Manchester Syntax. OWL is also a rather expressive language with many features the discussion of which exceeds the scope of this guide. We recommend the OWL 2 Primer~\cite{owlprimer} for a more in-depth description of the OWL language, its features and semantics. 
Instead, we will focus on those parts of the OWL language that occur most frequently in OWL ontologies. The ontology depicted in Figure~\ref{fig:vehicle-ontology} can be formalised in Manchester syntax as shown in Listing~\ref{lst:vehicle-omn} in the Appendix.
  
