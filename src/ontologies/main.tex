\section{Ontology development}

There is a wealth of literature on the subject of ontologies. These range from purely philosophical perspectives to very application-oriented ones from computer science. One of the reasons for this is that ontologies are used in a wide range of disciplines. Probably the best-known modern definition of the concept of ontology comes from Gruber\cite{gruber1993ontology,studer1998knowledge}: \textit{``An ontology is a formal, explicit specification of a shared conceptualization.''}. Although this definition has proved to be very suitable over the years, it is not very helpful if you want to gain a basic, intuitive understanding of ontologies. Therefore, we use a newer, more helpful definition of the term ontology in this guide:

\begin{definition}[Ontology \cite{neuhaus2018ontology}]
    An ontology of a given domain of interest is a document
that provides
\begin{enumerate}
    \label{def:ontology_2}
    \item  a \underline{vocabulary} for describing the domain of interest,
    \item  \underline{annotations} that document the vocabulary, and
    \item  a \underline{logical theory} (consisting of axioms and definitions) for the vocabulary,
in a way that these three elements together enable a competent user of the
ontology to ascertain its intended interpretation.

\end{enumerate}
\end{definition}

\paragraph{Vocabulary}

In this chapter we will explain these three components and their respective roles. The part of an ontology that most people should be familiar with is the \textit{vocabulary}. This collects the terms that are relevant for describing the target domain. Ontologies distinguish here fundamental categories:
\begin{enumerate}
    \item \textbf{Classes}: A class represents a general concept in an ontology, these can be categories of actual tangible things like "Dog" and also more abstract concepts like the growth process of a plant.
    
    \item \textbf{Individuals}: Individuals are specific members of classes within the ontology. They represent real-world entities like "my dog Bob" or "the marriage between Michelle and Barrack Obama".
    
    \item \textbf{Relationships}: Relationships define the connections or associations between individuals. They specify how different individuals are related to each other. I may be in the "owner of" relation with "my dog Bob".
\end{enumerate}

Ontologies are normally used for the abstract description of a domain. Concrete things (individuals) are usually not relevant for this purpose. Accordingly, ontology development usually focusses on the definition of classes and relations. For the ontology of pets, the term "Dog" is essential - but my dog Bob is not relevant for the description of the pet domain.

\paragraph{Annotations}

A collection of relevant terms is a good starting point for an ontology. However, this does not yet provide any information about the intended meaning of these terms. Therefore, all parts of the ontology are provided with annotations that document them. For example, each class should be annotated with a natural language definition.\todo{FN: In my opinion this is too strong. Some classes are not possible to define. } But annotations can also fulfil other purposes. For example, they can be used to give classes different readable names. It is therefore possible to use several names for the same concept, for example to reflect the names of a concept in different communities.\todo{FN: Also different languages (e.g/, English, German)} It is also often helpful to record certain meta-information with the help of annotations, such as the authors of a certain concept in the ontology.

\paragraph{Logical Theory} 

An annotated vocabulary is already a good source of information for understanding a domain - it is basically a domain-specific lexicon. However, it does not record the concrete relationships between different concepts. Although it is possible to describe these in textual annotations, there is a risk that these annotations will contradict each other. It is therefore important to record the connections between the concepts and relations in the ontology in a way that can be automatically checked for consistency. For example, it is helpful to express general axioms such as "Every dog is a mammal", "Every car has wheels" or "Renewable energy only comes from renewable sources". In order to formalize these axioms properly, a logical language is required. Although there are many possible languages that can be used for this purpose, the \textit{Web Ontology Language} (OWL) has established itself as the standard, which we will discuss in more detail in Section~\ref{ssec:ontology-languages}.

Figure~\ref{fig:vehicle-ontology} depicts a small example ontology with five classes and a single relationship (\textit{has part}). The class "Boat" is annotated with a definition. There are five \textit{is-a} relations depicted by white arrow heads and one axiom that defines that every car has some wheels.
\todo{FN: The definition of "boat" is wrong, since it includes ships, submarines, amphibious cars, all of which are disjoint with boat}
\begin{figure}
    \centering
    \begin{tikzpicture}
        \begin{pgfonlayer}{nodelayer}
            \node [style=class] (0) at (-7.25, 8.5) {Material Entity};
            \node [style=class] (1) at (-3.75, 5.75) {Wheel};
            \node [style=class] (2) at (-7.75, 5.75) {Car};
            \node [style=class] (7) at (-8.75, 7.25) {Vehicle};
            \node [style=class] (8) at (-10, 5.75) {Boat};
            \node [style=annotation, align=left] (13) at (-13, 7.25) {\small \textit{Definition:} A vehicle \\ \small that can travel in \\ \small water.};
        \end{pgfonlayer}
        \begin{pgfonlayer}{edgelayer}
            \draw [style=is-a] (1) to (0);
            \draw [style=rel] (2) -- (1) node[midway, above] {has part};
            \draw [style=is-a] (2) to (7);
            \draw [style=is-a] (7) to (0);
            \draw [style=is-a] (8) to (7);
            \draw [style=annot-edge] (13) to (8);
        \end{pgfonlayer}
    \end{tikzpicture}
    \caption{A simple ontology for the domain of vehicles.}
    \label{fig:vehicle-ontology}
\end{figure}

\subsection{The Web Ontology Language}
\label{ssec:ontology-languages}

The Web Ontology Language (OWL) is the most widely used formal language for ontology development. Our introduction of ontologies aligns closely with the structure of OWL. The main building blocks of OWL are logical axioms that define classes and their relationships. The OWL specification defines a number formats that can be used to develop and exchange ontologies. The default exchange format for OWL is its RDF/XML syntax. This syntax is however rather unwieldy and rather hard to read for most people. In this guide, we will therefore use the more readable OWL Manchester Syntax. OWL is also a rather expressive language with many features the discussion of which exceeds the scope of this guide. We recommend the OWL 2 Primer~\cite{owlprimer} for a more in-depth description of the OWL language, its features and semantics. 
Instead, we will focus on those parts of the OWL language that occur most frequently in OWL ontologies. The ontology depicted in Figure~\ref{fig:vehicle-ontology} can be formalised in Manchester syntax as shown in Listing~\ref{lst:vehicle-omn} in the Appendix.
  

\subsection{Quality criteria}

% \todo[inline]{Add quality criteria for ontologies}
% --> Tried to do so, hope you are fine with it, Markus

In ontology development, quality criteria are used to assess the effectiveness, reliability, and overall value of an ontology. There is a number of metrics that are based on different aspects of the ontology, like depth, number of axioms etc. For a more technical consideration of the quality of ontologies, conceptual models were developed for their assessment \cite{WilsonQuality2021, WilsonQuality2023}. In practical ontology work, however, we found the usefulness of concrete metrics to be limited. Ontologies are developed with a specific goal in mind and should also be evaluated against them. These goals may be as heterogeneous as the domains that they describe. Therefore, we will limit the discussion in this section to general aspects of ontology quality and encourage ontology developers to follow them as closely as possible. %These criteria support in ensuring that the ontology accurately represents the intended domain, is useful for its intended purpose, and can be easily reused or extended. The key quality criteria commonly used in assessing ontologies are:

\paragraph{Consistency}

The most important property of an ontology is its logical consistency. Consistency ensures that the ontology does not contain contradictions or conflicts within its structure or content. Depending on the formalisms used to develop the ontology, different mechanisms to endure consistency can be employed. Our workflow is based on the semantics of OWL DL. This allows the use of automated reasoners such as Hermit, Pellet and others to ensure consistency automatically.

\paragraph{Scope}

As we pointed out in Section~\ref{ssec:development-rules}, a scope definition is an essential part of good ontology development. Its purpose is to limit the kinds of concepts that are part of the ontology. But this scope definition can also function as a measure for the completeness of an ontology. Completeness assesses the extent to which the ontology captures the relevant concepts, relationships, and properties within its scope. Since an ontology is supposed to cover all essential aspects of a domain without omitting important information, assessing its completeness involves checking for gaps or missing elements that could affect usefulness.

\paragraph{Fitness for purpose}

An ontology should be build for a specific set of use-cases. These use-cases should be formalised in a way that they reflect the requirements of its intended stakeholders and applications. This can be done by defining specific modelling tasks that can then be used to measure the degree to which an ontology fits the purpose for which it was designed.

Competency questions are also a good indicator for the fitness of an ontology. These questions define inferences that the ontology should allow. These questions are orthogonal to the use-cases described before in that they evaluate the semantic quality of the ontology.
%It involves assessing whether the ontology effectively supports knowledge representation and reasoning tasks within its domain. Usefulness also considers factors such as performance, scalability, and positive user experience.

\paragraph{Agreement}

Open ontologies are meant to be used by a wider range of experts. It is therefore imperative that these experts understand the ways in which different concepts of the ontology can be used. An ontology that has a deep philosophical foundation and strong axiomatisation is still not usable, if external experts cannot understand the intended semantics of a given class from its definition, axioms and annotations. This can be assured by easily accessible documentation, and decision guides. BFO, for example, is based on complex definitions that are not immediately helpful for someone new to the field of formal ontologies. It does, however, come with an associated book and a helpful flow chart \cite{bernabe2023method} that make it more accessible.

\paragraph{Open development}

An ontology will never be perfect. There will always be concepts that are not covered or relationships that have not been axiomatieed. An ontology does therefore require constant maintenance that cannot always be provided by the initial developers. An open development workflow is therefore mandatory for an ontology that is usable in active, evolving fields. The ontology must therefore be developed in a way that allows the contribution of people external to the core development team in order to accommodate changes in the domain or requirements of the ontology as well as clear guidelines and documentation for ontology integration and extension%Maintainability also includes considerations such as versioning, governance, and ontology lifecycle management.

\paragraph{Reusability}

An ontology that is not used is a dead ontology. This is not just limited to using the ontology terms for specific annotations. An ontology may also be used as the foundation of another ontology. This is, in fact, necessary as each ontology should have a limited scope, whilst scientific domains in particular are often intertwined. It is therefore important to design the ontology in a way that it can be easily integrated with other ontologies. A reusable ontology should be modular and interoperable, allowing components to be easily reused in various contexts. This is specifically true for higher-level, such as top- and mid-level ontologies. 

%By evaluating ontologies against these quality criteria, developers can ensure that their ontologies are robust, reliable and fit for purpose.



%\todo[inline]{oof, using techniques of software development to asses the quality of ontologies is certainly a choice and I am not sure that it is a good one. I propose that we are very cautious here. Developing ontologies the same way one would develop software causes some serious issues. As someone with a background in computer science: Been there, done that, everything exploded.\\
%\cite{WilsonQuality2021, WilsonQuality2023}, however, seems to have some good points. In particular the use of patterns and identification of anti-patterns is something that people are curretnly working on in PMD.
%}

% Accordingly, the exploration of quality models in software engineering reveals several established frameworks that can inform and guide software development practices. McCall’s \cite{McCallModel1978}, Boehm’s \cite{BoehmQuality1976}, Garvin’s \cite{GarvinQuality1984}, ISO/IEC 25012 \cite{ISO25012}, and ISO 9126 \cite{ISO9126-2, ISO9126-3, ISO9126-4} are among the well-accepted models. Each of these offers unique perspectives and criteria for assessing software quality. Hence, these models provide hierarchies of characteristics and measures that enable comprehensive evaluations from various angles.
% Similarly, various quality models have been proposed in the field of ontology development. These models, such as the semiotic metric suite \cite{BurtonQuality2005}, Gangemi's model \cite{GangemiOntoEval2006}, and OQuaRE \cite{DuqueOQuaRE2011}, focus on assessing ontology quality based on 
% \begin{itemize}
%     \item structural, 
%     \item functional, and 
%     \item usability
% \end{itemize} \textit{dimensions}. While some models focus intrinsic aspects, others consider both intrinsic and extrinsic factors while providing a holistic view of ontology quality. \todo{Which of these employed by successful, large-scale ontologies?}

% Concerning characteristics and measures relevant to ontology evaluation, fourteen (14) significant characteristics have been identified in the literature, including:
% \begin{itemize}
%   \item Compliance,
%   \item Complexity,
%   \item Internal Consistency,
%   \item Modularity,
%   \item Conciseness,
%   \item Coverage,
%   \item External Consistency,
%   \item Comprehensibility (i.e., intrinsic point of view),
%   \item Accuracy,
%   \item Relevancy,
%   \item Functional Completeness,
%   \item Understandability (i.e., extrinsic point of view),
%   \item Adaptability, and
%   \item Efficiency.
% \end{itemize}
% These characteristics were grouped into the four \textit{evaluation aspects}, also defined as \textit{dimensions}, (i) Structural intrinsic, (ii) Domain intrinsic, (iii) Domain extrinsic, and (iv) Application extrinsic, that mirror the ontology evaluation space. This model resembles the ISO 25012 Data Quality model. Based thereon, three main categories for ontology quality can be defined: inherent ontology quality, domain-dependent ontology quality, and application-dependent ontology quality. Additionally, typical characteristics such as currentness, credibility, accessibility, availability, and recoverability also denoted in ISO 25012 can be used for ontology assessment. \todo[inline]{This guide is meant to pose as one-stop overview on how to develop ontologies. Referencing a large number of papers without an extensive explaination or discussion seems contrary to that goal. I propose that we pick one or two well-established tools (e.g. OOPS!) and explain their criteria}

\subsection{Before Development}

\subsubsection{Define the Kind of Ontology}
\label{sssec:ontologykind}

In order to develop a suitable ontology, it must first be determined what purpose it should actually fulfil. Different types of ontologies are required for different purposes. In general, three different types of ontologies are distinguished here, which differ fundamentally in the breadth of coverage and specificity of the defined terms.
The most general here are the so-called \textit{top-level ontologies}. These deal with fundamental distinctions of the world. They often define very abstract and philosophical concepts, e.g. material and immaterial objects or things with and without temporal extension. These ontologies are rarely used on their own in applications. Instead, they should be utilised so that new ontologies can use them as a basis on which to build.

The next level is formed by the so-called \textit{domain ontologies}. These deal with the description of a specific domain, e.g. the domain of materials science, the domain of copper production and processing, the domain of ultra-thin film technology or the domain of energy systems technology. As should already be clear from the examples, these domains can vary in size and also overlap. It is therefore particularly important that a new ontology fits well into the context of existing ontologies.

The \textit{application ontologies} are even more specific. These deal with the description of a very specific use case, e.g. to document a data format. These ontologies often only cover a very small domain and are highly specialised. For this purpose, they often import important, general concepts from several domain ontologies.

In this guide, we will mainly focus on domain ontologies. Top-level ontologies require in-depth knowledge and experience in ontology development. However, this guide is aimed at those who are new to the field of ontology development.


\begin{example}
Alice is an expert in the field of aerospace engineering and wants to develop a new ontology for her domain. In particular, she aims to integrate data sources from different airplane manufacturers. This task is limited to a particular domain and therefore the ontology should not be a top-level ontology. At the same time, she does not want to represent a particular application. Her task requires her to integrate dittetent views that experts haveonthe same particular domain.

\todo{example for application ontology}

\end{example}

\subsubsection{Define the scope}


After the target domain and level of detail of the ontology were identifies in the previous step \ref{sssec:ontologykind}, one must now precisely define the intended. A written \textit{scope definition} is created for this purpose. This is necessary because domains rarely exist in a vacuum. Accordingly, it is often the case that during development it becomes apparent that the description of a certain term actually requires concepts that lie just outside the scope of the ontology or even in a different domain. If these concepts are then simply added to the ontology and this procedure is often repeated, the result is an ontology that is no longer clearly structured, but instead contains a collection of concepts that are unrelated to the domain. This problem is also referred to as \textit{scope creep}. A fixed scope definition can counteract this by only including concepts in the ontology that fall within the definition.

This approach has several decisive advantages. Firstly, it allows users to quickly decide whether a term they are looking for could be found in a given ontology. It also reduces the risk of duplication of concepts across different ontologies, as each ontology remains clear in its focus and does not also contain additional "external" concepts that cause unexpected overlaps. A scope definition also makes it easier to find and reuse existing ontology resources, as the written scope definitions can be processed and indexed by search engines.

\begin{example}

Alice collects the requirements for her domain. In doing so,she realises that her main focus lies on the description of the manufacturing process for airplanes. She then formulates the scope of her ontology as follows: "This ontology defines relevant terms for the domain of aircraft production. This includes parts of aircrafts, processes that are used to assemble these parts. It does not include terms related to the aerodynamic properties of parts or planes, flight behaviours, description of chemical processes etc."

\end{example}

\subsubsection{Define Competency Questions}

One of the difficult tasks in ontology development is not the structuring of the ontology, but the choice of topics to focus on during development. A helpful tool for this is the definition of so-called \textit{competency questions}. These questions are a collection of questions that should be answerable using the ontology.

\begin{example}
\end{example}



\subsection{Ontology development}

In the previous sections, we introduced the main ingredients that need to be present in order to. The main focus of this document is, however, the process with which a useful ontology can be developed. Most ontologies are build for the purpose of represent of knowledge from different domains of interest or to integrate data from different sources. An ontology, therefore, serves as a shared conceptualisation between different actors or stakeholders. It is often wrongly assumed that this conceptualisation is something that already exists and can be extracted from analysing literature and domain descriptions. In practice, however, it quickly becomes apparent that these descriptions are imprecise at best and inconsistent at worst. Both cases are problematic if an ontology is to be used in practice to annotate new information. An imprecise ontology will be interpreted differently by different annotators, which in turn leads to inconsistent annotations. Instead, consensus finding is an essential step of good ontology development.

In this section, we will lay out the ingedients that we deem necessary for developing a good and usable ontology.

\subsubsection{Use a Top-Level Ontology}

Ontology as a philosophical discipline is concerned with the question what things are. However, projects that are concerned with the description of a particular domain do not really need to develop a general conceptualisation of reality. Yet , developing such a framework is very usefull to prevent early mistakes that will only show their severe effects late in development. Top-level ontologies model fundamental distinctions of reality. \todo{FN: Please elaborate}

One of the most widely used top-level ontologies is the Basic Formal Ontology (BFO, \cite{bfo-book}). It stands at the core of numerous ontologies and projects from the domain of bio-chemistry, energy and material science. BFO is well-documented and there are 

\subsubsection{Identify Important Concepts and Relations Within Scope}

\subsubsection{Reuse Where Possible}

\subsubsection{Aim For Good Definitions}

\subsubsection{Reach out to other experts}

The more general the ontology is, the more important it is to maintain an open view of the things to be modeled. An application ontology that is only focussed on a single task and will never be used beyond that scope, does not require much consensus with other experts. But as soon as the ontology is supposed to be used beyond its circle of developerts, e.g. by external data annotaters, it is crucial to develop an ontology that those external stakeholders would agree to.

* Engaging other experts helps with adoption

%It is crucial to ensuring that the ontology accurately reflects the knowledge and semantics of the domain. A consensus on aspects relevant to the domain leads to a shared understanding among stakeholders. This is essential for creating an ontology that is widely accepted and used within the community. Furthermore, achieving consensus helps in standardizing the representation of concepts and relationships within the ontology. This, in turn, promotes interoperability by ensuring that different systems and applications can understand and exchange information using a common semantic framework. Therefore, a consensus on ontological terms just ensures its usability. Ontologies aim to reduce ambiguity by providing a formal and explicit representation of knowledge. Consensus building helps clarify ambiguous terms and concepts, making the ontology more precise and reliable. Moreover, ontologies are often used to facilitate communication between humans and machines. A consensus-driven ontology ensures that the terminology and semantics used in the ontology align with the mental models of the domain experts, making it easier for humans to understand and contribute to the knowledge representation.

\subsubsection{Aim for Reuseability}

The design of an ontology encourages the development of other reusable and modular ontologies. A widely accepted ontology in a domain, such as mid-level or higher-level domain ontologies, can serve as a foundation for other related ontologies, fostering a more efficient and scalable knowledge representation ecosystem. By involving domain experts and stakeholders in the development process, potential errors, inaccuracies, or oversights can be identified and corrected early on, for which a consensus finding procedure may be necessary.

In general, building consensus involves engaging the community and incorporating diverse perspectives. This not only enhances the richness of the ontology but also promotes community involvement and ownership, leading to increased adoption and sustainability.
Ontologies need to be adaptable to changes in the domain over time. Hence, a flexible foundation that can evolve with the changing needs and understanding of the domain, ensuring the ontology remains relevant, has to be implemented, which can be found in consensus finding processes.

\subsubsection{Seperate Terminological and Factual Knowledge}

Ontologies are useful to capture general knowledge of a domain. As mentioned earlier, "Every dog is a mammal" may be a useful axiom for some domains, while the fact that "Bob is a dog" is probably not useful for a larger domain. In knowledge representation, these two statements fall into seperate categories:
The first statement belongs to the group of general statement about general concepts and their relations, also called the terminological box (TBox). The TBox encapsulates a collection of terminological axioms that delineate hierarchical relationships, classes, and overarching concepts pertinent to the domain. These axioms articulate subsumption relationships between classes, utilizing object properties to denote connections between them. Thus, they are representing a taxonomy or partial order among the classes. The TBox encompasses various types of axioms, including subsumption, equivalence, disjointness, and role hierarchy axioms which provides the structural underpinning for knowledge organization and facilitates automated reasoning to deduce new facts predicatable on the relationships.

Conversely, statement that "Bob is a dog" is a statement about a particular and as such is usually considered part of the assertional box (ABox). The ABox embodies the assertional knowledge regarding individual entities within the domain. It records assertions pertaining to the membership of instances in specific classes and the relationships existing between these instances. Typically, ABox statements comprise concept assertions, denoting an individual's membership in a particular class, and role assertions, elucidating relationships between individuals. By accommodating the representation of concrete facts and instances, the ABox enables the knowledge base to delineate specific scenarios or manifestations of concepts outlined in the TBox. For instance, a tangible object like a machine present in a laboratory can be digitally depicted as an instance within the ABox and linked to other instances making use of object properties defined in the TBox.

The synergy between the TBox and ABox provides a robust and formal framework for knowledge representation and inference. This coalescence empowers systems to execute logical inferences, conduct consistency evaluations, and address intricate queries within a specified domain. Thereby, comprehensive knowledge management and reasoning capabilities are facilitated. An ontology is usually part of the TBox. The terms of the ontology is then liked to individuals or data items in the ABox, which can be represented in various ways such as knowledge graphs or relational databases.



\subsection{Tools for ontology development}

\begin{description}
    \item[Robot] Robot is a useful tool to process, translate ontologies, extract modules and information from and reason with ontologies.\\ \url{http://robot.obolibrary.org/}
    \item[(py-)horned-owl] The rust library horned-owl and its python interface py-horned-owl are more modern interfaces to a general interface for ontology processing. \\ \url{https://github.com/phillord/horned-owl} and \\ \url{https://github.com/jannahastings/py-horned-owl}
    \item[{\protege}] {\protege} is a free, open-source ontology editor and framework used in research and industry for creating and managing ontologies. It provides a user-friendly, configurable user interface and supports various ontology languages and formats such as OWL/XML, RDF/XML, and Turtle. \\ \url{https://protege.stanford.edu/}
    \item[OntoPanel] Ontopanel is a plugin for the open-source graphical editor diagrams.net that facilitates ontology creation and editing. It includes a set of pipeline tools to foster ontology development on visual, graphical basis, comprising imports and reusage of ontologies, converting diagrams to OWL, verifying diagrams using OWL rules, and mapping data. \cite{ChenOntopanel2022} In particular, it is designed to support domain experts in ontology development. \\ \url{https://github.com/BAMresearch/Ontopanel-backend}
\end{description}
