The proposed ontology workflow uses alphanumeric identifiers in order to adhere to the development rules described in Section~\ref{ssec:development-rules}. These numerical identifier encode two pieces of information: A unique id for the given entity and the developer that added this entity.

Alphanumeric identifiers are listed as best
practice and commonly used because in most cases their advantages
outweigh the disadvantages. A short summary is given below:
\paragraph{Advantages} 
Versioning and backward compatibility: a name for a concept
can change in a new version of the ontology without its identifier (in
databases etc) having to be changed everywhere. - user groups: if people
talk about the same concept but can't decide on a main name for it they
can both be added as labels for this concept and the people can get
their own views of the ontology showing their preferred label. - better
readability of long names: in labels we dont have to use CamelCase. Long
names like NonRenewableMunicipalWasteFuel are better readable in lower
case as ``non renewable municipial waste fuel''.

\subsection{List of user ID's}
\label{ssec:list-of-user-ids}

The list of user ids can be found in the main ontology file
``''. The ID's got added in front of the \emph{dc:contributor}
annotation of the person they belong to. Please add yourself via pull
request.

\section{Motivation}\label{motivation}

See the
\href{https://github.com/OpenEnergyPlatform/ontology/issues/133}{\#133}
for the whole discussion. 

\hypertarget{disadvantages}{%
\subsubsection{Disadvantages}\label{disadvantages}}

\begin{itemize}
\tightlist
\item
  the file is harder to read without a tool like protege (that will
  automatically show labels instead of the alphanumerical identifiers).
\end{itemize}

\hypertarget{implementation-in-the-oeo}{%
\section{Implementation in the OEO}\label{implementation-in-the-oeo}}

Identifiers look like this: OEO\_00140123

They are structured in three parts: OEO\_{[}yyyy{]}{[}xxxx{]} -
``OEO\_'' identifies them as part of the OEO ontology - {[}yyyy{]}
identifies the user who added the classes. Each user gets a unique ID
(list below). Entities with the ID 0000 were created before the change
to alphanumeric identifiers was made. - {[}xxxx{]} identifies the
specific class added by the user. This enables every user to create
10000 entities before having to get a new ID. IDs are assigned
chronologically to each entity, starting with 0000.

\hypertarget{how-to-change-your-settings}{%
\section{How to change your
settings}\label{how-to-change-your-settings}}

Every user has to set his protégé user settings according to the
structure of the identifiers used in the OEO.

This is done by opening the ``oeo.omn'' file in Protégé and clicking on
\texttt{File\ -\textgreater{}\ Preferences\ -\textgreater{}\ new\ entities}.

Then the following settings are selected:

\texttt{Entity\ IRI:}\strut \\
- \texttt{Specific\ IRI:} http://openenergy-platform.org/ontology/oeo/ -
\texttt{/} - \texttt{Auto-generated\ ID}

\texttt{Entity\ Label:} - \texttt{Same\ as\ label\ renderer}

\texttt{Auto-generated\ ID:} - \texttt{Numeric\ (iterative)} -
\texttt{prefix:\ OEO\_{[}your\ user\ id{]}\ e.g.\ OEO\_0001\ if\ you\ user\ ID\ is\ 1}
- \texttt{Digit\ count:\ 4} - \texttt{Start:\ 0} - \texttt{End:\ -1}

check \texttt{"Remember\ last\ ID\ between\ Protégé\ sessions"}

then click ``OK''

\includegraphics{https://user-images.githubusercontent.com/56925445/79339344-086d7800-7f29-11ea-804f-1b4e2df3e6d5.png}
