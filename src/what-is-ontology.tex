\section{Understanding ontology development}

\subsection{What is ontology development?}
The term 'ontology development' refers to the process of creating and refining ontologies. An ontology is a formal and explicit representation of knowledge that defines the concepts within a domain and the relationships between them. It serves as a structured framework for organizing information and knowledge, facilitating understanding and sharing of data within a particular community and across different systems. Key components of ontologies include:

\begin{enumerate}
    \item \textbf{Classes}: A class represents a general concept in an ontology, these can be groups of actual physical objects like "Dog" and also more abstract concepts like the growth process of a plant.
    
    \item \textbf{Individuals}: Individuals are specific members of classes within the ontology. They represent real-world entities like "my dog Bob" or "the marriage between Michelle and Barrack Obama".
    
    \item \textbf{Properties}: Relationships define the connections or associations between individuals. They specify how different individuals are related to each other. I may be in the "owner of" relation with "my dob Bob".

    \item \textbf{Axioms}: Axioms are statements that express the rules or constraints governing the relationships and properties of concepts in the ontology. They help ensure consistency and coherence within the knowledge representation. They allow us to represent the general structure of a domain, e.g. "Every owner of a dog is a human."

\end{enumerate}

The ontology design process typically involves collaboration among domain experts, knowledge engineers, and other stakeholders. It may go through iterative processes of refinement and validation to ensure that the ontology accurately reflects the domain it is intended to represent. This may be referred to as 'curation'. Ontologies are used to enhance data interoperability and support automated reasoning.

The Semantic Web, in particular, relies heavily on ontologies to enable machines to understand and process information in a more meaningful way, fostering improved communication and integration of data across diverse applications and platforms. As such, ontologies are the basis of the Semantic Web and thus, they are essential for the generation of structured data and their interoperability.


\subsection{Finding Consensus}

Consensus finding in the field of ontology development refers to the process of reaching an agreement or shared understanding among stakeholders, domain experts, and other relevant parties involved in defining and constructing an ontology for a specific domain. It is a crucial step in ensuring that the ontology accurately reflects the knowledge and semantics of the domain, and it involves resolving differences in perspectives, terminology, and conceptualizations. As such, the process is inherently connected to ontology development and has to be performed inevitably.

In this section, some reasons for the importance and benefits of consensus finding in ontology development are presented.
A consensus on aspects relevant in ontology creation leads to a shared understanding of the domain among the stakeholders. This is essential for creating an ontology that is widely accepted and used within the community. Furthermore, achieving consensus helps in standardizing the representation of concepts and relationships within the ontology. This, in turn, promotes interoperability by ensuring that different systems and applications can understand and exchange information using a common semantic framework. Therefore, a consensus on ontological terms just ensures its usability. Ontologies aim to reduce ambiguity by providing a formal and explicit representation of knowledge. Consensus building helps clarify ambiguous terms and concepts, making the ontology more precise and reliable. Moreover, ontologies are often used to facilitate communication between humans and machines. A consensus-driven ontology ensures that the terminology and semantics used in the ontology align with the mental models of the domain experts, making it easier for humans to understand and contribute to the knowledge representation.
The design of an ontology encourages the development of other reusable and modular ontologies. A widely accepted ontology in a domain, such as mid-level or higher-level domain ontologies, can serve as a foundation for other related ontologies, fostering a more efficient and scalable knowledge representation ecosystem. By involving domain experts and stakeholders in the development process, potential errors, inaccuracies, or oversights can be identified and corrected early on, for which a consensus finding procedure may be necessary.

In general, building consensus involves engaging the community and incorporating diverse perspectives. This not only enhances the richness of the ontology but also promotes community involvement and ownership, leading to increased adoption and sustainability.
Ontologies need to be adaptable to changes in the domain over time. Hence, a flexible foundation that can evolve with the changing needs and understanding of the domain, ensuring the ontology remains relevant, has to be implemented, which can be found in consensus finding processes.
