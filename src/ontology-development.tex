
\section{Ontology development guidelines}

\subsection{Development Rules}
\label{ssec:development-rules}
For the consistent development of ontologies, a few basic rules and principles have to be followed.
Therefore, a selected and adapted list of principles based on the development principles\footnote{\url{https://obofoundry.org/principles/fp-000-summary.html}} of OBO Foundry\cite{smith2007obo} is used:

\begin{itemize}
    \item \textbf{Open} - The ontology MUST be openly available to be used by all without any constraint.
    \item \textbf{Common Format} - The ontology is made available in a common formal language in an accepted concrete syntax.
    \item \textbf{URI/Identifier Space} - Each ontology MUST have a unique IRI in the form of a permanent URL. \\ \textit{For the PMDco ontologies family, the namespace https://w3id.org/pmd/co is used}
    \item \textbf{Versioning} - The ontology provider has documented procedures for versioning the ontology, and different versions of ontology are marked, stored, and officially released.
    \item \textbf{Scope} - The scope of the ontology and content that adheres to that scope is clearly specified. \\ \textit{For the PMDco ontologies family, the scopes are defined HERE [INSERT URL?]}
    \item \textbf{Textual Definitions} - The ontology has textual definitions for its classes.
    \item \textbf{Documentation} - The owners of the ontology should strive to provide as much documentation as possible. \\ \textit{For the PMDco ontologies family, comprehensive documentation can be found in the corresponding {\github}  repository: \url{https://github.com/materialdigital/core-ontology}}
    \item \textbf{Collaboration} - Ontology development should be carried out in a collaborative fashion.
    \item \textbf{Authority} - There should be an authority responsible for communications between the community and the ontology developers, for mediating discussions involving maintenance in the light of scientific advance, and for ensuring that all user feedback is addressed.
    \item \textbf{Naming Conventions} - The names (primary labels) for elements (classes, properties, etc.) in an ontology must be intelligible to scientists and amenable to natural language processing.
    \item \textbf{Responsiveness} - Ontology developers shall offer channels for community participation and be responsive to requests.
\end{itemize}

\subsection{Ontology Design Patterns}

The PMD design process is informed by the PMD Semantic Patterns. These patterns describe general design patterns that can be used to model similar concepts in a similar fashion. 

Example(s): 
\begin{itemize}
    \item OTTR\footnote{\url{https://ottr.xyz/}}
\end{itemize}

\subsection{Usage of external ontologies}

Example(s): 
\begin{itemize}
    \item ChEBI\footnote{\url{https://www.ebi.ac.uk/chebi/}}
    \item QUDT\footnote{\url{https://qudt.org/}}
\end{itemize}

\section{The {\github} workflow}

Ontology development unfolds across two primary platforms: {\github} issues and online meetings. {\github}  serves as the central hub, where issues act as the primary conduit for external feedback, extension requests, and developmental discussions. Online meetings are strategically employed for tackling challenging and contentious definitions. The overarching objective in both {\github}  and meetings is to reach a consensus among all participating experts, ensuring a robust and collaborative ontology development process.

In the following, we propose a general workflow from a user request to the final extension of the ontology.

An example for this workflow and instructions on how to create your own can be found in our example repository.\footnote{\url{https://github.com/open-ontology-community/example-ontology-repository}}

\subsection{Reporting missing classes/properties/axioms}

% \todo{MS: Add a note on the extend of the issue topic: ideally create atomic issues (1 term = 1 issue), however, this is sometimes not feasible, because some terms and relations have to be discussed in context with others. for a single issue max a couple (1-5) of related terms/relations. In OEO we also use meta-issues for broader topics, that can be split / refer to atomic issues.}

Ontology development is never truly done. There will always be some fragments of the domain that have not yet been included. Likewise, you may bring a new perspective to the domain that does not match with the ontologie's current state. In both cases, you are welcome to contribute your feedback to the ontology development. To do so, you need to open a new issue in the respective {\github} repository of this ontology. 

We recommend a few simple rules for how issues should be structured:

\begin{enumerate}
    \item \textbf{Naming:} Name the issue in such a way that the issue title makes sense if read after ``The issue is that ...''
    \begin{itemize}
        \item Good: ``Microscopes are not covered'', ``The definition of iron is inconsistent''
        \item Bad: ``Can we cover microscopes?'', ``faulty definition of iron''
    \end{itemize}
    \item \textbf{Keep it simple:} Issues should cover minimal topics. In the best case that means that they only address a single term. Ontology development, however, is inherently complex and terms can often not be discussed in a vacuum. To keep things manageable, we recommend to limit issues to at most five concepts.
    \item \textbf{Supply information:} Try to be as precise in your description of the change that you want to see. Ontology development is about concepts, not terms. So, a simple statement that a particular term is missing from the ontology does not suffice. Instead, try to describe your understanding of that term and - in the best case - propose a draft definition for your term. 
\end{enumerate}

\begin{example}
    Alice notices that microscopes are not yet covered by any of the PMD ontologies. She opens a new issue titled \textit{``Microscopes are not covered yet''}. She also proposes a new definition \textit{``A microscope is a tool that makes things bigger''}. As she submits, the issue tracker assigns the issue number \textit{3141}.
\end{example}

\subsection{Working on open issues}

The ontology development takes place in {\github} issues. The main goal of these issues is to collect different opinions on the matter at hand and find a solution that reflects a consensus among all experts.

\begin{enumerate}
    \item \textbf{Check scope: } A lead developer evaluates whether the new concept fits into the scope definition of the ontology
    \item \textbf{Assign:} Assign a single ontology developer (further called \textit{assignee}) who is responsible for the implementation of this new concept.
    \item \textbf{Draft:} The assignee drafts a definition of the new concept and possible subsumption relations and axioms.
    \item \textbf{Discuss:} Other developers give feedback on these proposals and contribute to the discussion until a consensus is reached.
    \item \textbf{Agree or Talk:} If the discussion exceeds {\issuediscussionlimit} posts, it is moved to online meetings
\end{enumerate}

\begin{example}
    Betty is one of the lead developers. They receive a notification of Alice's class request. They review the request and note that the concept is indeed missing from the current version of the ontology and within scope. They then decide that Claire will become the main assignee.

    Claire notes that the proposed definition is not precise enough, as it would also include other machines like tensile testing machines. She therefore proposes an alternative definition: \textit{``A microscope is a tool that magnifies objects.''}

    Dillan, another ontology developer, points out that 'magnifying' is already in the ontology and that microscopes should be in relation to that class: 
    \[\mathrm{`microscope`} \sqsubseteq \exists \mathrm{`participates~in`}.\mathrm{`magnifying`}\]

    Claire points out that a microscope does not necessarily need to participate in a process in order to be a microscope and a discussion ensues. At this point, the discussion exceeds the threshold of {\issuediscussionlimit} posts and they decide to discuss this matter in the developer meeting and add issue number 1314 to the agenda.
\end{example}

\subsection{Development meetings}

The issue-based workflow should allow developers to address the majority of the feedback from users, stakeholders and other developers. However, some concepts are so integral to the domain or the distinctions for specific concepts are so complex or controversial, that finding a consensus is a tough task. Experience shows that an online setting with long response times is not the most efficient way to deal with such issues and that face-to-face discussions turn out to be much more productive.

Therefore, we recommend regular developer meetings to address these tough issues - be it in-person or online via Zoom or similar services. 

\begin{example}
In the next developer meeting, Betty, Claire and Dillan revisit issue 1314. During the discussion, they conclude that it is a function of microscopes to be used in magnifying processes. Therefore they decide to use the definition \emph{``A microscope is an object that can be used to magnify other objects''} and add the axiom
\emph{```microscope` `has~function` \textbf{some} \`magnifying~function`''}.
The experts agree that they have reached a consensus and document their discussion in issue 1314. Dillan is tasked with the implementation of this issue.
\end{example}

\subsection{Implementing a change}
\begin{itemize}
    \item  Work should be done in a separate branch names ``[IssueNumber]"
    \item add issue number to PR, use "closes \# issue number" to make {\github} close the issue once the PR is merged
    \item update changelog and/or term tracking
\end{itemize}

\begin{example}
Dillan has been tasked with implementing the consensus reached in the developer meeting. To do so, he pulls the most recent version from the development of the ontology and creates a new branch named \textit{1314\_add\_microscope}.

Then Dillan adds the necessary axioms to the ontology. He chooses to do so using {\protege}. He did already set up his working environment according Section~\ref{ssec:using-protege}. He opens the ontology in {\protege} and adds the missing class including its definition and the additional axiom.

Afterwards, he pushes his new branch to {\github} and opens a new Pull Request to merge his branch into the development branch.
\end{example}

\subsection{Reviewing a change}
Review the changes made on {\github} in the "files changed" section or commit-wise. Mark small change requests in the code directly or write a comment. Further, pull the proposed changes from the feature branch to protege and check there for inconsistencies. 

The following checklist can guide the review:
\begin{itemize}
    \item[$\Box$] Are classes and axioms placed correctly in the hierarchy?
    \item[$\Box$] Are spelling and grammar of definitions as well as axioms correct?
    \item[$\Box$] Are class definitions also reflected in the class axioms?
    \item[$\Box$] Are the classes classified correctly according to the BFO?
    \item[$\Box$] Are changes placed in the correct file, i.g. in case of a modular structure?
    \item[$\Box$] Are all steps of the requested changes done according to the issue and the expert decision?
    \item[$\Box$] Are there only changes that were previously agreed on in the discussion?
    \item[$\Box$] Are all changes made intentional? (e.g.Sometimes Protégé rearranges things which goes wrong in rare cases)
    \item[$\Box$] Are all relevant changes mentioned in the changelog / term trackers?
\end{itemize}
    
\begin{example}
Once Dillan is done with the implementation and pushed his proposed changes to {\github}, he marks the PR as "ready for review" and assigns Betty and Eric, a domain expert, as reviewers. 

Both start the review process by checking the questions of the reviewer's guide (list above). 
Betty has a change request, because Dillon missed to add "microscope" to the list of added terms in the changelog and marks this in the PR. Eric is content with the implementation and approves the PR, after Dillon added the note the the changelog.

After the approval and after the automated checks (CI) in {\github} were successful, Dillon merges the PR to the development branch. The term "microscope" has been added to the ontology.
\end{example}

\paragraph{Structure of the repository}

\begin{description}
    \item[modules] Contains the actual ontology modules (T-box)
    \begin{itemize}
        \item Ontology files in Turtle or Manchester Syntax
    \end{itemize}
    \item[patterns] Contains PMD Semantic Patterns that can be used to structure the ontology development and the shape definition
    \begin{itemize}
        \item LinkML
        \item SHACL
        \item OTTR
    \end{itemize}
    \item[shapes] General shapes that can be used to describe certain scenarios
    \begin{itemize}
        \item SHACL
    \end{itemize}
\end{description}

\section{Tools for ontology development}

\begin{description}
    \item[Robot] Robot is a useful tool to process, translate ontologies, extract modules and information from and reason with ontologies. \url{http://robot.obolibrary.org/}
    \item[(py-)horned-owl] The rust library horned-owl and its python interface py-horned-owl are more modern interfaces to a general interface for ontology processing \url{https://github.com/phillord/horned-owl} and \url{https://github.com/jannahastings/py-horned-owl}
    \item[{\protege}] \url{https://protege.stanford.edu/}
\end{description}

\subsection{Using {\protege} for ontology development}

The proposed ontology workflow uses alphanumeric identifiers in order to adhere to the development rules described in Section~\ref{ssec:development-rules}. These numerical identifier encode two pieces of information: A unique id for the given entity and the developer that added this entity.

Alphanumeric identifiers are listed as best
practice and commonly used because in most cases their advantages
outweigh the disadvantages. A short summary is given below:
\paragraph{Advantages} 
Versioning and backward compatibility: a name for a concept
can change in a new version of the ontology without its identifier (in
databases etc) having to be changed everywhere. - user groups: if people
talk about the same concept but can't decide on a main name for it they
can both be added as labels for this concept and the people can get
their own views of the ontology showing their preferred label. - better
readability of long names: in labels we dont have to use CamelCase. Long
names like NonRenewableMunicipalWasteFuel are better readable in lower
case as ``non renewable municipial waste fuel''.

\subsection{List of user ID's}
\label{ssec:list-of-user-ids}

The list of user ids can be found in the main ontology file
``''. The ID's got added in front of the \emph{dc:contributor}
annotation of the person they belong to. Please add yourself via pull
request.

\section{Motivation}\label{motivation}

See the
\href{https://github.com/OpenEnergyPlatform/ontology/issues/133}{\#133}
for the whole discussion. 

\hypertarget{disadvantages}{%
\subsubsection{Disadvantages}\label{disadvantages}}

\begin{itemize}
\tightlist
\item
  the file is harder to read without a tool like protege (that will
  automatically show labels instead of the alphanumerical identifiers).
\end{itemize}

\hypertarget{implementation-in-the-oeo}{%
\section{Implementation in the OEO}\label{implementation-in-the-oeo}}

Identifiers look like this: OEO\_00140123

They are structured in three parts: OEO\_{[}yyyy{]}{[}xxxx{]} -
``OEO\_'' identifies them as part of the OEO ontology - {[}yyyy{]}
identifies the user who added the classes. Each user gets a unique ID
(list below). Entities with the ID 0000 were created before the change
to alphanumeric identifiers was made. - {[}xxxx{]} identifies the
specific class added by the user. This enables every user to create
10000 entities before having to get a new ID. IDs are assigned
chronologically to each entity, starting with 0000.

\hypertarget{how-to-change-your-settings}{%
\section{How to change your
settings}\label{how-to-change-your-settings}}

Every user has to set his protégé user settings according to the
structure of the identifiers used in the OEO.

This is done by opening the ``oeo.omn'' file in Protégé and clicking on
\texttt{File\ -\textgreater{}\ Preferences\ -\textgreater{}\ new\ entities}.

Then the following settings are selected:

\texttt{Entity\ IRI:}\strut \\
- \texttt{Specific\ IRI:} http://openenergy-platform.org/ontology/oeo/ -
\texttt{/} - \texttt{Auto-generated\ ID}

\texttt{Entity\ Label:} - \texttt{Same\ as\ label\ renderer}

\texttt{Auto-generated\ ID:} - \texttt{Numeric\ (iterative)} -
\texttt{prefix:\ OEO\_{[}your\ user\ id{]}\ e.g.\ OEO\_0001\ if\ you\ user\ ID\ is\ 1}
- \texttt{Digit\ count:\ 4} - \texttt{Start:\ 0} - \texttt{End:\ -1}

check \texttt{"Remember\ last\ ID\ between\ Protégé\ sessions"}

then click ``OK''

\includegraphics{https://user-images.githubusercontent.com/56925445/79339344-086d7800-7f29-11ea-804f-1b4e2df3e6d5.png}


\section{Quality criteria}

% \todo[inline]{Add quality criteria for ontologies}
% --> Tried to do so, hope you are fine with it, Markus

In ontology development, quality criteria are used to assess the effectiveness, reliability, and overall value of an ontology. These criteria support in ensuring that the ontology accurately represents the intended domain, is useful for its intended purpose, and can be easily reused or extended. The key quality criteria commonly used in assessing ontologies are:

\begin{itemize}
    \item \textbf{Correctness}: This criterion evaluates the accuracy and precision of the ontology. It involves ensuring that the concepts, relationships, and axioms within the ontology accurately reflect the real-world domain being modeled. Correctness also involves verifying that logical inferences and deductions possibly made by using the explicit knowledge given in the ontology are consistent and valid.

    \item \textbf{Completeness}: Completeness assesses the extent to which the ontology captures all relevant concepts, relationships, and properties within the domain regarded. Since an ontology is supposed to cover all essential aspects of a domain without omitting important information, assessing its completeness involves checking for gaps or missing elements that could affect usefulness.

    \item \textbf{Consistency}: Consistency ensures that the ontology does not contain contradictions or conflicts within its structure or content. It involves verifying that there are no conflicting statements or redundant information present in the ontology. Consistency also ensures that the ontology adheres to defined modeling conventions and standards.

    \item \textbf{Clarity}: Clarity refers to the readability and understandability of the ontology. It involves ensuring that the ontology is well-organized, properly documented, and uses clear and unambiguous terminology. A clear ontology should be easy for users to navigate and comprehend to facilitate effective communication and knowledge sharing.

    \item \textbf{Reusability}: Reusability assesses the extent to which the ontology can be applied to different contexts or integrated with other systems. A reusable ontology should be modular, adaptable, and interoperable, allowing components to be easily reused in various applications. However, this is specifically true for higher-level, such as top- and mid-level ontologies. Reusability also involves providing clear guidelines and documentation for ontology integration and extension (cf. \textbf{Clarity}).

    \item \textbf{Usefulness}: Usefulness evaluates the practical utility and value of the ontology for its intended stakeholders and applications. It involves assessing whether the ontology effectively supports knowledge representation and reasoning tasks within its domain. Usefulness also considers factors such as performance, scalability, and positive user experience.

    \item \textbf{Maintainability}: Maintainability assesses the ease of updating, managing, and evolving the ontology over time. It involves evaluating flexibility, extensibility, and ability to accommodate changes in the domain or requirements of the ontology. Maintainability also includes considerations such as versioning, governance, and ontology lifecycle management.
\end{itemize}

By evaluating ontologies against these quality criteria, developers can ensure that their ontologies are robust, reliable and fit for purpose.

For a more detailed consideration of the quality of ontologies, conceptual models were developed for their assessment \cite{WilsonQuality2021, WilsonQuality2023}. These are based on quality models in the field of software engineering. 
Accordingly, the exploration of quality models in software engineering reveals several established frameworks that can inform and guide software development practices. McCall’s \cite{McCallModel1978}, Boehm’s \cite{BoehmQuality1976}, Garvin’s \cite{GarvinQuality1984}, ISO/IEC 25012 \cite{ISO25012}, and ISO 9126 \cite{ISO9126-2, ISO9126-3, ISO9126-4} are among the well-accepted models. Each of these offers unique perspectives and criteria for assessing software quality. Hence, these models provide hierarchies of characteristics and measures that enable comprehensive evaluations from various angles.
Similarly, various quality models have been proposed in the field of ontology development. These models, such as the semiotic metric suite \cite{BurtonQuality2005}, Gangemi's model \cite{GangemiOntoEval2006}, and OQuaRE \cite{DuqueOQuaRE2011}, focus on assessing ontology quality based on 
\begin{itemize}
    \item structural, 
    \item functional, and 
    \item usability
\end{itemize} \textit{dimensions}. While some models focus intrinsic aspects, others consider both intrinsic and extrinsic factors while providing a holistic view of ontology quality.

Concerning characteristics and measures relevant to ontology evaluation, fourteen (14) significant characteristics have been identified in the literature, including:
\begin{itemize}
  \item Compliance,
  \item Complexity,
  \item Internal Consistency,
  \item Modularity,
  \item Conciseness,
  \item Coverage,
  \item External Consistency,
  \item Comprehensibility (i.e., intrinsic point of view),
  \item Accuracy,
  \item Relevancy,
  \item Functional Completeness,
  \item Understandability (i.e., extrinsic point of view),
  \item Adaptability, and
  \item Efficiency.
\end{itemize}
These characteristics were grouped into the four \textit{evaluation aspects}, also defined as \textit{dimensions}, (i) Structural intrinsic, (ii) Domain intrinsic, (iii) Domain extrinsic, and (iv) Application extrinsic, that mirror the ontology evaluation space. This model resembles the ISO 25012 Data Quality model. Based thereon, three main categories for ontology quality can be defined: inherent ontology quality, domain-dependent ontology quality, and application-dependent ontology quality. Additionally, typical characteristics such as currentness, credibility, accessibility, availability, and recoverability also denoted in ISO 25012 can be used for ontology assessment.

\label{ssec:using-protege}